% \begin{abstract}
% \section*{}
\label{sec:intro}

\textbf{tl;dr: Byte pair encoding for fuzzy variables}

A robot learning to navigate the world finds that the
combination of
certain sensors gives more information than either of them alone.
The combination of $x$- \textit{and} $y$-position tells more about
when to watch for obstacles than either variable on its own.
The speed \textit{and} position of arm tells more about what shoulder
torque needs to be generated than either variable on its own.
These cases
of sensor interaction can be hard coded manually. Human designers
often exploit their knowledge of the system to do this
via feature engineering,
but in cases where the robot system is too complex
to intuit these useful interactions, they can be learned.
Automatically creating these predictive features is the goal of 
the \textbf{Ziptie} algorithm.

(If you want to skip the How and Why of it, jump to
\textbf{\hyperref[sec:code]{using the Ziptie library}}.)

Ziptie makes a uncommon assumption that all sensor signals,
$\alpha_i$, (otherwise known as \textbf{features}) are
\textbf{\hyperref[sec:fuzzy]{Fuzzy variables}} with
$\alpha_i \in [0, 1]$. It also assumes that
a fixed number of features, $n$, are
received at discrete time intervals in a vector $\mathbf{A}$.
\begin{equation}
\mathbf{A} = (\alpha_1, \alpha_2, \alpha_3, ..., \alpha_n)
\end{equation}
The $n$ features of the sensor data
can be imagined as $n$ separate 
\textbf{cables}, $\phi_i$, each carrying a single signal
(as they often are in robots). The
challenge of clustering these cables into informative combinations
can then be imagined as creating \textbf{bundles} of cables, $\phi_{ij}$,
as with a ziptie.

A new bundle is created when the \textbf{agglomeration energy},
$\gamma_{ij}$
between two cables exceeds a threshold, $C_\gamma$. Agglomeration energy is the
accumulated \textbf{coactivation}, $\kappa_{ij}$, of the cable pair, where the
coactivation at each time step is given by the product of their two
activities.
\begin{equation}
\kappa_{ij} = \alpha_i \alpha_j
\end{equation}
Once bundle $\phi_{ij}$ is created, it gets first dibs at representing any
amount of signal carried on both cables $\phi_i$
and $\phi_j$.
\begin{equation}
\alpha_{ij} = \min(\alpha_i, \alpha_j)
\end{equation}
The member cables of $\phi_{ij}$  retain only the residual signal.
\begin{eqnarray}
\alpha_{\hat{i}} &=& \alpha_i - \alpha_{ij}\\
\alpha_{\hat{j}} &=& \alpha_j - \alpha_{ij}
\end{eqnarray}
This approach constrains bundles' activities to remain on [0, 1] as well.
Bundles can be coactive with cables and with other bundles.
Any cable-cable, cable-bundle, or bundle-bundle pair whose
agglomeration energy exceeds $C_\gamma$ will nucleate
a new bundle.

As Ziptie continues to operate on the stream of cable activities,
the total number of bundles will continue to grow. As bundles are 
bundled again
together, the number of cables in the largest bundles will grow too.
These can be limited from growing too large by introducing
a \textbf{stopping condition}, such as
a maximum number of bundles or fixed number of time steps.

% The rest of this paper attempts to answer the questions you might have:
% \begin{itemize}

% \item{What's \textbf{\hyperref[sec:background]{all this}} about?}

% \item{What are \textbf{\hyperref[sec:fuzzy]{Fuzzy variables}}
% and why do they matter?}

% \item{How does the \textbf{\hyperref[sec:algorithm]{Ziptie}}
 %algorithm work in practice?}

% \item{How can I use \textbf{\hyperref[sec:code]{Ziptie code}}?}

% \item{How is Ziptie related to what
% \textbf{\hyperref[sec:bio]{biological brains}} do?}


% \end{itemize}

% \end{abstract}
