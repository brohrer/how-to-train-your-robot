% \begin{abstract}
% \section*{}
\label{sec:intro}

A robot learning to interpret its sensors and world, finds that the
combination of
certain sensors gives more information than either individually.
The combination of x- \textit{and} y-position sensors tells more about
the when to watch for obstacles than either on its own.
The speed \textit{and} position of arm tells more about what shoulder
torque needs to be generated that either on its own. Often these cases
of sensor cooperation can be added manually, through human designers'
knowledge of the system, but in cases where the robot is too complex
for this, they can be learned. This is the goal of 
\textbf{Agglomerative Continual Dimensional Clustering (ACDC)}.

ACDC makes a non-traditional assumption that all signals are
\textbf{\hyperref[sec:fuzzy]{Fuzzy Categorical variables}}, that is,
that their values vary between
zero and one, inclusive. It assumes a fixed number of signals that are
received at discrete time intervals.

equation n-dim

ACDC the n dimensions of the signal can also be conceived as n separate 
\textbf{cables}, each carrying a single signal (as they often are in robots). The
challenge of clustering these cables into informative combinations
can then be conceived as creating \textbf{bundles} of cables, as with
a \textbf{ziptie}.

A new bundle are created when the \textbf{agglomeration energy}
between two cables exceeds a threshold. Agglomeration energy is the
accumulated \textbf{coactivation} of the cable pair, where the
coactivation at each time step is given by the product of their two
activities.

% equation coactivity_{ab} = activity_a * activity_b

Once bundle ab is created, it gets first dibs at representing any
signal carried on cable a and cable b.

% equation bundle_{ab}_activity = minimum(activity_a, activity_b)

It's member cables retain only the residual signal.

% equation activity_a_residual = activity_a - bundle_{ab}_activity
% equation activity_b_residual = activity_b - bundle_{ab}_activity

Bundles' activities will also remain on the interval [0, 1].
They can also be coactive with cables and with other bundles.
Any cable-cable, cable-bundle, or bundle-bundle pair whose
agglomeration energy exceeds the chosen threshold nucleate
a new bundle.

The number of cables in a bundle and the total number of bundles are
not limited. Given sufficient time they can continue to increase.
This can be limited by a \textbf{stopping condition}, such as
a maximum number of bundles or fixed number of time steps.



% \end{abstract
