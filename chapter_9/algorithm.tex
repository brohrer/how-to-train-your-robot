\section{How the Ziptie Algorithm Works}
\label{sec:algorithm}

How it works, step-by-step.

\subsection{Why coactivation matters}
\label{subsec:whycoactivation}

Feature agglomeration was mentioned in Section~/ref{subsec:featureagg}.
Feature agglomeration groups features based on how often
they have similar values. It tries to group nearly identical
features first. Ziptie on the other hand groups features based on how
often they are co-active. (When they are both zero, that doesn't
increase their similarity.) This creates groups of features that
are simultaneously active, things that
happen at the sime time. While this will capture features that are identical,
it will also capture unrelated features whose co-occurence gives
valuable information.
